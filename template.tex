%!TEX TS-program = xelatex
%!TEX encoding = UTF-8 Unicode

\documentclass[$fontsize$, a4paper]{article}

% LAYOUT
%--------------------------------
% Margins
\usepackage{geometry}
\geometry{$geometry$}

% Do not indent paragraphs
\setlength\parindent{0in}

% Enable multicolumns
\usepackage{multicol}
\setlength{\columnsep}{10pt}

% Uncomment to suppress page numbers
% \pagenumbering{gobble}

% LANGUAGE
%--------------------------------
% Set the main language
\usepackage{polyglossia}
\setmainlanguage{$lang$}

% TYPOGRAPHY
%--------------------------------
\usepackage{fontspec}
\usepackage{xunicode}
\usepackage{xltxtra}
% converts LaTeX specials (quotes, dashes etc.) to Unicode
\defaultfontfeatures{Mapping=tex-text}
\setromanfont [Ligatures={Common}, Numbers={OldStyle}]{$mainfont$}
% Cool ampersand
\newcommand{\amper}{{\fontspec[Scale=.95]{$mainfont$}\selectfont\itshape\&}}

% MARGIN NOTES
%--------------------------------
\usepackage{marginnote}
\newcommand{\note}[1]{\marginnote{\scriptsize #1}}
\renewcommand*{\raggedleftmarginnote}{}
\setlength{\marginparsep}{7pt}
\reversemarginpar

\def\changemargin#1#2{\list{}{\rightmargin#2\leftmargin#1}\item[]}
\let\endchangemargin=\endlist

% HEADINGS
%--------------------------------
\usepackage{sectsty}
\usepackage[normalem]{ulem}
\sectionfont{\rmfamily\mdseries}
\subsectionfont{\rmfamily\mdseries\scshape\normalsize}
\subsubsectionfont{\rmfamily\bfseries\upshape\normalsize}

% PDF SETUP
%--------------------------------
\usepackage{hyperref}
\hypersetup
{
  pdfauthor={$name$},
  pdfsubject={$name$'s CV},
  pdftitle={$name$'s CV},
  colorlinks, breaklinks, xetex, bookmarks,
  filecolor=black,
  urlcolor=[rgb]{0.117,0.682,0.858},
  linkcolor=[rgb]{0.117,0.682,0.858},
  linkcolor=[rgb]{0.117,0.682,0.858},
  citecolor=[rgb]{0.117,0.682,0.858}
}

% DOCUMENT
%--------------------------------
\begin{document}

{\LARGE $name$}\\[.2cm]
{\large \textsl{DevOps / System Engineer}}\\[.1cm]

% CONTACT

\begin{multicols}{2}

$for(phone)$
$phone$
$endfor$

\href{mailto:$email$}{$email$}\\

\columnbreak

For detailed list of projects visit:\\
\href{http://linkedin.com/in/$linkedin$}{linkedin.com/in/$linkedin$}

% Born: 1987\\
% Nationality: Polish\\
% Martial status: married\\

\end{multicols}


% \vspace{10pt}

% INTRO
$intro$

% \vspace{5pt}

% SKILLS

\begin{multicols}{2}

\subsection*{Competencies}
$for(skill)$
\emph{\#} \enspace $skill$\\
$endfor$

\columnbreak

\subsection*{Areas of Interest}
$for(interests)$
\emph{\#} \enspace $interests$\\
$endfor$

\end{multicols}


% EDUCATION

\begin{multicols}{2}

\section*{Education}
$for(education)$
$education.year$\\
\emph{$education.institute$}\\
\textbf{$education.subject$}, $education.degree$\\

$endfor$

\columnbreak

\section*{Cerification}

\begin{itemize}
  \setlength\itemsep{-0.5em}
  \item \href{https://www.redhat.com/rhtapps/certification/verify/?certId=140-054-446}{Red Hat Certified Engineer}
  \item \href{https://www.redhat.com/rhtapps/certification/verify/?certId=140-054-446}{Red Hat Certified System Administrator}
\end{itemize}

\section*{Languages}
\begin{itemize}
  \setlength\itemsep{-0.5em}
  \item \emph{Polish} (Native)
  \item \emph{English} (C1)
  \item \emph{French} (A2)
\end{itemize}

\end{multicols}

\vspace{-10pt}


% EXPERIENCE

\section*{Experience}
\noindent

% CERN

\note{04/2015--Present}
\textbf{\textsc{CERN} (Geneva)}\\
\emph{System Engineer}

\vspace{10pt}
(1) \emph{System Administration Modernisation}
\begin{itemize}
  \item Support and troubleshooting of over 2000 OpenStack VMs, desktops and servers critical to the operation of CERN accelerators.
  \item Transformation of an old bash/awk/Python script-based configuration management to a modern Ansible-based solution.
  \item Replacement of inconsistent/non-existing version controlling of various parts of system configuration with a single git tree.
  \item Massive analysis, comparison, fixing errors and making more consistent of kickstart files for 2000+ machines (included also changes in PXE servers and generating kickstart files based on Ansible inventory file and CERN's internal sources of data about hosts).
  \item Implementing a filesystem monitoring system based on auditd, rsyslog, Elasticsearch, Logstash/Logalike and Kibana (dashboards).
  \item Creating monit configuration for process management and monitoring.
  \item Developing dozens of Ansible playbooks for configuration of various aspects of OS and applications.
\end{itemize}

% \vspace{10pt}
\pagebreak
(2) \emph{Process Management and Deployment (PoC)}
\begin{itemize}
  \item Design and implementation of the platform that could replace the former legacy service management solution,
  while keeping the interface and release-deploy-configure workflow for users backward compatible.
  \item Development of Python and Ansible-based tool for:
  \begin{itemize}
    \item deployment of the accelerators controls applications;
    \item adding/removing the services to/from an applications database (to make them persistent across OS reinstallations);
    \item generating configuration for service monitoring and management (based on monit);
    \item providing a consistent solution for supplying user specific (monit) checks for a service or group of services.
  \end{itemize}
\end{itemize}

\vspace{10pt}
(3) \emph{Accelerator Project Exploitation Tools - Tracing}
\begin{itemize}
  \item System administration of Elasticsearch/Logalike/Kibana clusters:
  \begin{itemize}
    \item upgrade to ES 2.0 and Kibana 4;
    \item creation of nginx reverse proxy configuration;
    \item creation of monit configuration and checks for all ELK services;
    \item writing a Logalike extension in Java 8.
  \end{itemize}
\end{itemize}


\noindent\rule[0.5ex]{\linewidth}{1pt}

% Linux Polska
\note{02/2014--03/2015}\textbf{\textsc{Linux Polska} (Warsaw)}\\
\emph{Solution Architect}

\vspace{10pt}
(1) \href{http://opensourceday.com/}{Open Source Day}
\begin{itemize}
  \item Organization of the biggest Open Source event in Poland.
\end{itemize}

(2) EuroLinux
\begin{itemize}
  \item Design and implementation of Automated Delivery Pipeline for building RPM packages (using bash scripts chained into Jenkins' jobs):

  \begin{itemize}
  \item building RPM packages with open source tools (from Fedora Project):
  koji, mock, mash/createrepo, revisor/pungi, Sigul;
  \item maintenance and configuration of Jenkins, GitLab, git repositories, etc;
  \item development of bash and Python scripts;
  \item configuration of testing environment based on Vagrant, Packer, RHEV's virtual machines;
  \item set up of Zabbix monitoring.
  \end{itemize}

\end{itemize}


\noindent\rule[0.5ex]{\linewidth}{1pt}

% IMPAQ
\note{11/2011--01/2014}\textbf{\textsc{IMPAQ Group} (Warsaw)}\\
\emph{Software / System Engineer}
\begin{itemize}
  \item Support of RHEL/Oracle/Java telco applications for international cellular networks.
  \item Installation of a OpenStack cluster integrated with a GlusterFS cluster (automated with Chef).
  \item Development of Chef cookbooks for automatic orchestration of development and testing environments of Java web applications (+Apache Tomcat and Oracle XE 11g).
  \item Installation, configuration (using Chef) and maintenance of the Hadoop cluster and additional tools.
  \item Marketing and pre-sales activities for M2M, Cloud Computing and Big Data Business Practice.

  \item Design and implementation of the Linux services infrastructure in mVoip project:
  \begin{itemize}
    \item load balancing and HA mode: HAProxy and nginx;
    \item MySQL Master-Master Semisynchronous Replication cluster and Apache Cassandra cluster.
    \item Zabbix monitoring and syslog-ng centrallized logging;
    \item Jira OnDemand administration
  \end{itemize}

\end{itemize}

% \noindent\rule[0.5ex]{\linewidth}{1pt}

% Outbox
%\note{07/2011--10/2011}\textbf{\textsc{Outbox} (Warsaw)}\\
%\emph{Junior Consultant}
%\begin{itemize}
%  \item Development of the CRM system based on Oracle PeopleSoft platform for Telekomunikacja Polska SA.
%\end{itemize}

%\noindent\rule[0.5ex]{\linewidth}{1pt}

% ERSA
%\note{04/2015--04/2016}\textbf{\textsc{49th ERSA Congress} (Lodz)}\\
%\emph{Intern}\\
%Work mostly as a member of IT support and a member of information helpdesk during the Congress.

%\noindent\rule[0.5ex]{\linewidth}{1pt}

% fedorapl.org
%\note{04/2015--04/2016}\textbf{\textsc{fedorapl.org}}\\
%\emph{Team member (Voluntary)}\\
%Development of web portal actively supporting and providing forum, wiki and other facilities for Polish community of Fedora Linux users.

\end{document}
